\documentclass[a4paper,12pt]{scrartcl}
\usepackage{amssymb}

\begin{document}
\ptconcl[10cm]{Wir werden nur krank, wenn sich entweder Viren oder das Immunsystem ständig ändern.
-S-
Wir werden krank und Viren ändern sich nicht.
-C-
Also ändert sich das Immunsystem.}

\bigskip\ptconcl{\ptlogic{all x (~Kx -> ~(all y(Vy -> ~Cy) xor all y(Iy -> ~Cy)))}
-S-
\ptlogic{all y (Vy -> ~Cy)}
-S-
\ptlogic{all x Kx}
-C-
\ptlogic{~all y (Iy -> ~Cy)}}

\bigskip\ptkdns{
a, \ptlogic{all x (~Kx -> ~(all y(Vy -> ~Cy) xor all y(Iy -> ~Cy)))},, P
b, \ptlogic{all y (Vy -> ~Cy)},, P
c, \ptlogic{all x Kx},, P
d, \ptlogic{~Kx' -> ~(all y(Vy -> ~Cy) xor all y(Iy -> ~Cy))}, a&b, US
e, \ptlogic{(all y (Vy -> ~Cy) xor all y (Iy -> ~Cy)) -> Kx'}, d, AL
f, \ptlogic{Kx'}, c, US
g, \ptlogic{all y (Iy -> ~Cy)}, e&f, ??
}

\bigskip\noindent Die Konklusion konnte nicht aus den Prämissen abgeleitet werden, also ist der Schluss ungültig.

\bigskip\noindent\ptconcl{\ptlogic{~p -> ~(q xor r)}
-S-
\ptlogic{q}
-S-
\ptlogic{p}
-C-
\ptlogic{r}}

\bigskip\ptkdns{
a, \ptlogic{~p -> ~(q xor r)},, P
b, \ptlogic{q},, P
c, \ptlogic{p},, P
}
\end{document}